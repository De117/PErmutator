\documentclass[11pt,final,conference,a4paper]{IEEEtran}
\usepackage[utf8]{inputenc}
\usepackage[T1]{fontenc}
\usepackage[noadjust]{cite}
\usepackage{url}
\usepackage[dvips]{graphicx}

\begin{document}

\IEEEoverridecommandlockouts

\title{Fine grained permutation of text code in PE files}

\author{
	\authorblockN{Bruno Humić, Stjepan Groš}
	\authorblockA{
		Faculty of Electrical and Computing Engineering \\
		University of Zagreb\\
		Unska bb, 10000 Zagreb, Croatia \\
		E-Mail: bruno.humic@gmail.com, stjepan.gros@fer.hr}}

\maketitle

\begin{abstract}
Memory corruption is one of the oldest security vulnerabilities
that is still very common today. It allows attackers to gain full
control of victims computer. Crucial in the step of gaining control
is for attacker to know memory layout of a process. There are a
number of protection mechanisms developed to counter this threat,
one of which are randomization mechanisms that make process image
layout unpredictable for attacker. Randomization based security
mechanisms had a major impact on dealing with memory corruption
vulnerabilities but the threat still remains. In this paper we
describe work started on a new randomization security mechanism
whose goal is to rearrange memory blocks in the code section of a
PE file. In that way, we strive to achieve higher security levels
of a system by making process layout even more unpredictable for
the attackers.
\end{abstract}

\begin{keywords}
PE file format, computer security, operating systems, ASLR, DEP, ASLP, randomization, memory corruption, permutation, control flow graph, Microsoft Windows
\end{keywords}

\section{Introduction}
\label{sec:intro}

Species in the nature use diversity in order to combat different
threats. In other words, if a threat manages to infect one specimen
it want be able to infect many others because all of them are
different. In computer networks, when a threat compromises one
machine, it will probably compromies many others too because they
are all identical, or almost identical.

This is a know premise, and there are attempts to apply principles
from the nature in a computer networks. Yet, we are still far away
from a satisfactory solution.

The paper is structured as follows. First, we review related work
in Section \ref{sec:architecture}. We also list shortcomings of
existing approaches in that section. Then, in Section \ref{sec:design}
we describe the idea behind our solution, and its design and
implementation. In Section \ref{sec:testing} we give results of
different tests we performed on PErmutator. Finally, we conclude
paper in Section \ref{sec:conclusions}.

\section{Related work}
\label{sec:architecture}

\section{PErmutator design and implementation}
\label{sec:design}

Heuristics

Problems

PErmutaror Flow (graph generation, \ldots)

\section{Testing and validation}
\label{sec:testing}

\subsection{Validation}

Running vulnerable application to test if protection is better than ASLR.

\subsection{Performance}

Loading and executing application

Running application performance penalties

\section{Conclusions and Future Work}
\label{sec:conclusions}

\bibliographystyle{IEEEtran}
\bibliography{bibliography}

% Temporary fix until there are some references in the text
\nocite{arscryptolocker}

\end{document}
